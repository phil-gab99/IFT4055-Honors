\documentclass[final,3p,times,twocolumn,12pt,authoryear]{elsarticle}

\usepackage{amssymb}

\journal{Université de Montréal}

\begin{document}
\begin{frontmatter}
  \title{MQL to Cypher - A Framework for Translating a Domain-Specific Query
    Language for Querying Version-Controlled Modeling Repositories into Cypher
    Expressions}
  \author{Philippe Gabriel}
  \date{December 18, 2022}
  \ead{philippe.gabriel.1@umontreal.ca}
  \affiliation{
    organization={Université de Montréal},
    city={Montréal},
    state={Québec},
    country={Canada}
  }

  \begin{abstract}
    Version Control Systems (VCS) are responsible for managing changes and offer
    a plethora of other features over collections of information. Some of the
    most common VCSs such as \textit{git} are geared towards managing textual
    information which characterizes them as \textit{line-oriented}. Model Driven
    Engineering (MDE) is gaining more industrial interest and line-oriented VCSs
    are not the most appropriate for managing MDE projects as the interpretation
    of the changes in some serialized model over different versions of the
    affected artefact are not specific to the domain under which the model is
    conceived. A \textit{semantically-oriented} VCS is believed to better
    interpret and manage a model-driven project. Such a system needs to adhere
    to various components such as defining the units of comparison for semantic
    comparison, offer the necessary features for managing the repository, the
    ability to query the VCS with respect to the project, ... [\cite{dsvcs}]. In
    this paper we present a framework based on direct translations of MQL - a
    query language for querying a version-controlled modeling repository which
    corresponds to one of the prescribed components of a semantically-oriented
    VCS - into Cypher - a query language supported by Neo4j graph
    databases. Generated Cypher expressions are computed inside the database
    itself.
  \end{abstract}

  \begin{keyword}
    DSL \sep modelling \sep model transformation \sep cypher \sep neo4j \sep
    query language
  \end{keyword}

\end{frontmatter}

\section{Introduction}
Context of work, problem to solve.
Expose rest of paper.

% Multiple sections for solution
\section{Modelling the Repository}

\section{Related Work}
Sample text.

\section{Conclusion}
Sample text.

\bibliographystyle{elsarticle-harv} 
\bibliography{refs}

\end{document}